\documentclass[pdf, intlimits, unicode, S6, presentation]{beamer}
\usepackage[T2A]{fontenc}
\usepackage[cp1251]{inputenc}
\usepackage[english,russian]{babel}

\usetheme{Warsaw}
\usefonttheme[onlysmall]{structurebold}
\usefonttheme[onlymath]{serif}
\setbeamercovered{dynamic}
\setbeamercovered{transparent}
\setbeamerfont{institute}{size=\normalsize}
%\setbeamertemplate{background canvas}[vertical shading][bottom=red!10,top=blue!10]
\setbeamertemplate{theorems}[unnumbered]
\setbeamertemplate{footline}{}
\setbeamertemplate{headline}{}
\setbeamercolor{bluetext_color}{fg=blue}
\newcommand{\bluetext}[1]{{\usebeamercolor[fg]{bluetext_color}#1}}

\usepackage{beamerthemesplit}
\usepackage{dsfont}
\usepackage{pgfplots}
%\usepackage{graphics}

\newcounter{sli}
\newcommand{\slnumber}{\hfill {\color{lightgray} \refstepcounter{sli}\arabic{sli}}}

\title[ФЯиГ]
{Тестирование случайности кваканья лягушек}
\author{Курсовая работа \\
    студента 111 группы А.~В.~Пискунова}
\institute{{Саратовский государственный университет} \\
    им.~Н.~Г.~Чернышевского \\[5pt]
Кафедра математической кибернетики\\ и компьютерных наук\\[5pt]
Научный руководитель: к.~ф.-м.~н., доцент Придорожнова~Е.~А.
}
\date{2013г.}


\begin{document}

% Титульный лист - обязательный слайд
\begin{frame}
    \titlepage
\end{frame}

% Постановка задачи - обязательный слайд
\begin{frame}
    \frametitle{Постановка задачи \slnumber}
    \Large
    \begin{enumerate}
        \item Исследовать возможность использования кваканья лягушек в качестве датчика случайных чисел в полевых условиях;
        \item Провести сравнительную характеристику случайности кваканья для различных водоемов заданной местности.
    \end{enumerate}
\end{frame}


% Описание курсовой работы
\begin{frame}
    \frametitle{Район исследования  \slnumber}
    \Large
    Водоемы в районе сел Подгоренка и Свищёвка Ртищевского района Саратовской области:
    \begin{enumerate}
        \item Заливные озера бассейна реки Хопер: оз.~Кривое, оз.~Лебяжье;
        \item Заводь реки Вшивка;
        \item Деревенский пруд с.~Подгоренка.
    \end{enumerate}
\end{frame}

\begin{frame}
    \frametitle{Метод исследования  \slnumber}
    \Large
    \begin{enumerate}
        \item В качестве случайного числа брался период времени в нс между двумя последовательными кваканиями;
        \item Каждое случайное число приводилось к числу из диапазона $[0,1]$;
        \item Для набора случайных значений производилось вычисление числа $\pi$ с помощью метода Монте-Карло.
    \end{enumerate}
\end{frame}

\begin{frame}
    \frametitle{Метод Монте-Карло для вычисления числа $\pi$  \slnumber}
    \Large
    \begin{columns}
        \begin{column}{5cm}\centering
            \includegraphics{Pi-MonteCarlo}
        \end{column}
        \begin{column}{5cm}
            \[S=1\]
            \[S_c=\dfrac{\pi}4\]
            \[\pi = 4\frac {S_c}S\]
        \end{column}
    \end{columns}
    \vspace{1em}
    \Large
    \begin{description}
        \item[Количество точек:] 5000;
        \item[Количество повторений эксперимента:]  10.
    \end{description}
\end{frame}


\begin{frame}
    \frametitle{Результаты экспериментов  \slnumber}
    \centering
    Среднее отклонение вычисленного значения \\ от точного значения числа $\pi$
        
    \centerline{\small
        \begin{tikzpicture}
        \begin{axis}[
            width=10cm, height = 6.5cm,
            ybar, 
            yminorgrids, ymajorgrids,
            minor y tick num=5, 
            enlargelimits=0.25,
            ylabel={Отклонение},
            xlabel={Время проведения экспериментов},
            symbolic x coords={{4:00--9:00}, {9:00--12:00}, {12:00--18:00}, {18:00--5:00}},
            legend style={at={(0.5,-0.25)},anchor=north,legend columns=-2},
            nodes near coords align={vertical},
            ]
        \addplot coordinates {  ({4:00--9:00},      0.0000013) 
                                ({9:00--12:00},     0.0000011) 
                                ({12:00--18:00},    0.0000016) 
                                ({18:00--5:00},     0.0000016)};
        \addplot coordinates {  ({4:00--9:00},      0.0000001) 
                                ({9:00--12:00},     0.0000006) 
                                ({12:00--18:00},    0.0000005) 
                                ({18:00--5:00},     0.00000011)};
        \addplot coordinates {  ({4:00--9:00},      0.00000015) 
                                ({9:00--12:00},     0.0000012) 
                                ({12:00--18:00},    0.0000010) 
                                ({18:00--5:00},     0.00000019)};
        \addplot coordinates {  ({4:00--9:00},      0.00000011) 
                                ({9:00--12:00},     0.0000007) 
                                ({12:00--18:00},    0.0000009) 
                                ({18:00--5:00},     0.00000021)};
        \legend{пруд, р.~Вшивка, оз.~Кривое, оз.~Лебяжье}
        \end{axis}
        \end{tikzpicture}
    }
\end{frame}


% Описание результатов курсовой работы - обязательный слайд
\begin{frame}
    \frametitle{Результаты курсовой работы  \slnumber}
    \large
    Эксперименты показали:
    \begin{enumerate}
        \item в полевых условиях можно использовать кваканье лягушек в качестве датчика случайных чисел;
        \item на точность полученных данных влияет как время суток так и место расположения водоема;
        \item в качестве датчиков случайных чисел лучше всего подходит заводь реки Вшивка (около с.~Подгоренка) "--- для этого водоема получены самые стабильные результаты;
        \item самые нестабильные результаты получены на деревенском пруду в с.~Подгоренка.
    \end{enumerate}    
\end{frame}



% Выводы и перспектива дальнейшей работы - необязательно
\begin{frame}
    \frametitle{Возможные дальнейшие исследования  \slnumber}
    \Large
    \begin{enumerate}
        \item Необходимо более тщательно исследовать влияние удаленности водоема от населенных пунктов на точность результатов;
        \item Необходимо исследовать зависимость результатов измерений от породы лягушек;
        \item Нужно изучить влияние химического состава водоема и окружающей среды на возможность получения достоверных результатов измерений.
    \end{enumerate}
\end{frame}


% список литературы - обязательно
\begin{frame}
    \frametitle{Список использованных источников  \slnumber}
    \small
    \begin{thebibliography}{9}
        \setbeamertemplate{bibliography item}[article]
    	\bibitem{Liagushki-art}
            Д.~А.~Шабанов, С.~Н.~Литвинчук
    		\newblock
    		Зелёные лягушки: жизнь без правил или особый способ эволюции?
      		\newblock\emph{Природа}. "---
    		2010. "--- 
    		\No~3. "---
    		С.~29--36.
        \setbeamertemplate{bibliography item}[online]
        \bibitem{el-Rana-temporaria}
            \verb"http://www.ecosystema.ru/08nature/amf/35.htm"
            \newblock Травяная лягушка — Rana temporaria Linnaeus
        \setbeamertemplate{bibliography item}[article]
        \bibitem{Monte-Carlo-art}
            N.~Metropolis, S.~Ulam
            \newblock The Monte Carlo Method
            \newblock \emph{J. Amer. statistical assoc.}. "---
            2049. "--- 
            Vol.~44, \No~247. "---
            Pp.~335--341.
        \setbeamertemplate{bibliography item}[book]
        \bibitem{Monte-Carlo-book}
            J.~M.~Hammersley, D.~C.~Handscomb
            \newblock Monte Carlo methods
            \newblock Methuen: 1964. "---
            178~p.
        \setbeamertemplate{bibliography item}[online]
        \bibitem{el-Pi-Monte-Carlo}
            \verb"http://habrahabr.ru/post/128454/"
            \newblock Вычисление числа Пи методом Монте-Карло
    \end{thebibliography}
\end{frame}



\begin{frame}
	\begin{beamercolorbox}[ht=7ex, dp=4ex, center, shadow=true, rounded=true]{title in head/foot}
	\centerline{\LARGE{СПАСИБО ЗА ВНИМАНИЕ!}}
	\end{beamercolorbox}
\end{frame}

\end{document} 